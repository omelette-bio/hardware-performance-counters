\documentclass{rapport}
\usepackage[utf8]{inputenc}

\usepackage{pifont} % Pour les symboles appelés par la macro \ding
\usepackage{url} % Comme son nom l'indique, pour les url...
\usepackage{enumitem}
\usetikzlibrary{positioning} % Bibliothèque tikz pour positionner des nœuds relativement à d'autres

\usepackage[colorlinks, citecolor=red!60!green, linkcolor=blue!60!green, urlcolor=magenta]{hyperref} % Pour que les liens soient cliquables. Les options permettent de mettre les liens en couleur.

\usepackage{algorithm}
\usepackage{algo}
\usepackage{colorationSyntaxique}


% Pour un rapport en français 
% \usepackage[francais]{babel} % Commenter pour un rapport en anglais
% \renewcommand\bibsection{\section*{Bibliographie}} % Commenter pour un rapport en anglais

\englishTitlePage % Décommenter pour une page de titre en anglais


\pagestyle{fancy}
\renewcommand{\sectionmark}[1]{\markboth{\thesection.\ #1}{}}
\fancyfoot{}

\fancyhead[LE]{\textsl{\leftmark}}
\fancyhead[RE, LO]{\textbf{\thepage}}
\fancyhead[RO]{\textsl{\rightmark}}

\def\Latex{\LaTeX\xspace}
\def\etc{\textit{etc.}\xspace}



\title{Hardware Performance Counters}
\author{Francois Flandin}
\supervisor{Sid Touati}
\date{Second semestre de l'année 2024-2025}

\universityname{Université Côte d'Azur} % Nom de l'université.
\type{TER} % Type de document
\formation{Master Informatique} % Nom de la formation

% Retrouver les autres options possibles dans le document rapport.pdf

\begin{document}

	\maketitle

	\begin{abstract}
	Calmos le ramoloss, l'abstract c'est pas pour tout de suite
	\end{abstract}

	\clearpage
	\tableofcontents
	\clearpage
	\section{Introduction}
	If you wish to explore the details of the work done for the project, the project's code is hosted on GitHub at https://github.com/omelette-bio/projet-tutorat-s2-m1
	Parler ici de ce que sont les compteurs materiels de performance 
	Parler ici du but du projet 


	\clearpage

	\part{Checking Hardware Compatibility}

	In this part, we'll describe how to read precise informations about our CPUs, and go beyond the \texttt{cpuid} linux command. We'll also cover how to find the events mesurable by the CPU.

	\section{Checking CPU capabilities : the \texttt{CPUID} instruction}

	x86 architecture CPUs provide an instruction called \texttt{CPUID} that provides various informations on the CPU by reading special registers, for exemple, CPU name, memory address sizes, but also Hardware Performance Counters capacities.\newline
	The \texttt{CPUID} instruction is divided in \texttt{leafs} and \texttt{sub-leafs}, allowing to read multiple informations, stored statically in the CPU. These data are then stored in eax, ebx, ecx, and edx registers for the programmer to read.\newline
	For exemple, on intel, the leaf 0x80000008 stores in the eax register the physical address size and virtual address size of the CPU.\newline
	Intel and AMD manuals use a specific notation for this instruction which is the following : \texttt{CPUID[LEAF].REG} to more easily tell the leaf and register used for obtaining any information, with the previous exemple, such notation would give \texttt{CPUID[0x80000008].EAX}.

	\subsection{Using CPUID to find HPCs informations on INTEL CPUs}

	\texttt{CPUID[0x0A].EAX} gives us these informations
	\begin{description}[itemsep=1pt, parsep=0pt, topsep=0pt]
		\item [byte 1 {[7:0]}:] version number of the Architectural Performance Monitoring
		\item [byte 2 {[15:8]}:] number of general purpose PMC\footnotemark[1] per logical core 
		\item [byte 3 {[23:16]}:] bit size of the PMC registers 
		\item [byte 4 {[31:24]}:] number of architectural events
	\end{description}
	but, \texttt{CPUID[0x0A].EDX} gives us also the number of fixed PMC per logical core. \newline
	\newline Here are the results on the Intel test machine :
	\begin{verbatim}
		Performance Monitoring Version : 5
		Bit width of a PMC register
		Number of general purpose PMC per logical core : 8
		Number of fixed PMC per logical core : 4
		Number of architectural events : 8
	\end{verbatim}

	\footnotetext[1]{PMC : Performance Monitoring Counter}

	\subsection{Using CPUID to find HPCs informations on AMD CPUs}

	On AMD, it's \texttt{CPUID[0x80000001].ECX} that gives us informations about HPCs.
	\begin{description}[itemsep=1pt, parsep=0pt, topsep=0pt]
		\item [bit 10 :] support of IBS (specific to AMD, will be explained in part jsp combien)
		\item [bit 23 :] support of 6 Core Performance Counters.
		\item [bit 24 :] support of 4 NorthBridge Performance Counters.
		\item [bit 25 :] support of 4 L2 Cache Performance Counters.
	\end{description}

	\section{List mesurable events}

	blablabla....

	\part{HPCs on x86 processors}

	In this part will be explained how Hardware Performance Counters are implemented (non) and how to use them.
	Explainations hold for x86 processors, as it was verified on INTEL and AMD CPUs, I don't know if these informations hold for other architectures such as ARM for exemple.

	\section{Intel's specificities}

	\section{AMD's specificities}

	\part{Benchmarks}

	\section{Using the perf linux command}

	\section{Using the libpfm C-library}

	\pageblanche
	\bibliographystyle{apalike-fr}
	\bibliography{biblio}

\end{document}
